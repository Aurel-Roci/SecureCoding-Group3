\RequirePackage[l2tabu,orthodox]{nag}

% TODO: decide if one-sided/two-sided
%\documentclass[headsepline,footsepline,footinclude=false,fontsize=11pt,paper=a4,listof=totoc,bibliography=totoc,BCOR=12mm,DIV=12]{scrbook} % two-sided
\documentclass[headsepline,footsepline,footinclude=false,oneside,fontsize=11pt,paper=a4,listof=totoc,bibliography=totoc]{scrbook} % one-sided

\input{settings/packages}
\input{settings/settings}
\input{settings/commands}

% TODO: remove if glossary not needed
\input{glossary/terms}
\input{glossary/acronyms}

\newcolumntype{R}[1]{>{\raggedleft\let\newline\\\arraybackslash\hspace{0pt}}m{#1}}

\begin{document}

\input{pages/cover}

\frontmatter{}


{
	\chapter{Executive Summary}\
	We managed to fix the most of the crucial vulnerabilities of the web application.
	

}

\pagebreak
%\input{pages/title}
%\input{pages/disclaimer}
%\input{pages/acknowledgements}
%\input{pages/abstract}
\microtypesetup{protrusion=false}
\tableofcontents{}
\microtypesetup{protrusion=true}
\mainmatter{}

\chapter{Time Tracking Table}

\begin{table}[htb]

	\centering
	\resizebox{9cm}{!}{%
	\begin{tabular}{lll}
		\hline
		{\textbf{Name}} & { \textbf{Task}} & {\textbf{Time}} \\ \hline
		Aurel Roci & Test File Extensions Handling for Sensitive Information and documenting & 1\\
		& Test HTTP Methods and documenting & 1\\
		& Test HTTP Strict Transport Security and documenting & 1\\
		& Test RIA cross domain policy & 1\\
	    & Error Handling   &  0.5       \\
		& Testing for default credentials & 0.5  \\
		& Testing for Reflected Cross Site Scripting and documenting &  0.5  \\
		& Testing for Stored Cross Site Scripting and documenting& 0.5 \\
		& Testing for HTTP Verb Tampering and documenting &  0.5     \\
		& Testing for SQL Injection and documenting &  2 \\
		& Test Number of Times a Function Can be Used Limits & 0.5 \\
		& Test Business Logic Data Validation and documenting & 1 \\ 
		& Testing for Bypassing Session Management Schema and documenting & 1\\
		& Testing for Cookies attributes and documenting & 1 \\
		& Testing for Session Fixation and documenting & 1\\
		& Testing for Exposed Session Variables and documenting & 1\\
		& Testing for Cross Site Request Forgery and documenting & 1\\
		& Testing for logout functionality and documenting & 1\\
		& Test Session Timeout and documenting & 1\\
		& Testing for Session puzzling and documenting & 1\\
		& Testing Report & 2  \\
		& Testing for Cross Site Request Forgery  & 0.5  \\
		& Testing for Privilege Escalation  & 1  \\
		& Presentation & 0.25 \\
		\hline
		\hline
		Stefan Ch. Kofler & Reverse-Engineer the binary file & 10\\
		& Binary-equivalent & 9 \\
		& Decompile jar file & 1 \\
	 
		
	\end{tabular}%
}
\end{table}

%\input{chapters/01_introduction}
% TODO: add more chapters here

\chapter{Application Architecture} \


\chapter{Security Measures} \

\chapter{Fixes}

\subsection{Testing for Weak lock out mechanism(OTG-AUTHN-003)}\
After 3 wrong attempts to login the account now is blocked and an employee has now to approve it, for the user to be able to login. \

\textit{online\_banking/models/user.php} lines 177-178:\\

\begin{lstlisting} 
$query = "UPDATE users SET approved = 0 WHERE 
                 username='".mysql_real_escape_string($username)."';"; 
$result = mysql_query($query);
\end{lstlisting} 
After the user tries to log 3 times and fails the account will be blocked until an employee approves it again. So if the attacker deletes the session and tries again the account is blocked.\\


\subsection{Testing Directory traversal/file include (OTG-AUTHZ-001)}\

Fixed by fixing Command Injection and Local File Inclusion

\pagebreak
\subsection{Test Session Timeout(OTG-SESS-007)}\

Added a functionality in \textit{online\_banking/init.sec.php} in lines 17-21 :  


\begin{lstlisting}
(isset($_SESSION['LAST_ACTIVITY']) && (time() - $_SESSION['LAST_ACTIVITY'] > 900)) { 
session_unset();    
session_destroy();   
}
\end{lstlisting} 


This checks the last activity done in the website. If there is no activity for 15 minutes then the session is destroyed.\\



\subsection{Testing for Reflected Cross Site Scripting(OTG-INPVAL-001)} \

Insert a code in line 77 of the\textit{ online\_banking/employee.php} : 
\begin{lstlisting}
$search = htmlspecialchars($search);  

\end{lstlisting}
and line 54 of \textit{online\_banking/passwordReset.php}:  
\begin{lstlisting} 
htmlspecialchars($_GET['id']) 
\end{lstlisting}
to sanitize the input, so no code can be executed.
\pagebreak
\subsection{Testing for Stored Cross Site Scripting(OTG-INPVAL-002)}\
Insert code in lines 17 of \textit{online\_banking/customer.inc.php} to sanitized the input from XSS with the following code:   

\begin{lstlisting} 
$description = htmlspecialchars($description);
\end{lstlisting}

Changed the code in line 22 of \textit{online\_banking/register.inc.php} from :

\begin{lstlisting} 
    if(preg_match("/^$passwordRegex/", $password))
    to:
    if(preg_match("/^$emailRegex/", $email))
\end{lstlisting}
There was an if statement that checked for the password written twice, and we changed one to check for the email.\\



\subsection{Testing for SQL Injection (OTG-INPVAL-005) and Mysql testing (OTG-INPVAL-005)}\
Modified the query in line 131 of the \textit{online\_banking/models/user.php} file to sanitize the input: 
  
\begin{lstlisting} 
$query = "UPDATE users SET approved = 1 WHERE 
                       id='".mysql_real_escape_string($user_id)."';";
\end{lstlisting}
Also we sanitized the input in lines 137-140:

\begin{lstlisting}
$user_id = mysql_real_escape_string($user_id); 
$balance = mysql_real_escape_string($user_id);
$user_id = htmlspecialchars($user_id);
$balance = htmlspecialchars($balance);
\end{lstlisting}

This should prevent \textit{SQL Injections}.\\

\pagebreak
\subsection{Testing for Code Injection, Testing for Local File Inclusion, Testing for Remote File Inclusion(OTG-INPVAL-012)}\


Inserted a function when we read the variable in \textit{online\_banking/applicationDownload.php} line 2 :  

\begin{lstlisting} 
$user_id = intval($_GET['user_id']);
\end{lstlisting} 
to receive only integer values.
This will prevent any code injections.\\

\subsection{Testing for Command Injection(OTG-INPVAL-013)}\
Inserted code in line 3 of the \textit{online\_banking/applicationDownload.php} : 

\begin{lstlisting} 
$user_id = escapeshellarg($user_id);
\end{lstlisting}

 This way the shell commands are removed from the variable\\
 

\subsection{Testing for incubated vulnerabilities(OTG-INPVAL-015)}\
This vulnerability was fixed by fixing XSS, SQL Injection, Code Injection, Command Injection.
 
 
\subsection{Analysis of Error Codes (OTG-ERR-001)}\

Changed the PHP \textit{error\_reporting()} such that it does not display any errors: \textit{online\_banking/init.sec.php}; \textit{online\_banking/customer.sec.php}\\

\pagebreak
\subsection{Test Business Logic Data Validation(OTG-BUSLOGIC-001)}\

Added a server side check using regular expression in \textit{online\_banking/customer.sec.php} line 21: 
	\begin{lstlisting} 
	if($amount<0 || preg_match("/^$amountRegex/", $amount)) 
	\end{lstlisting}
	
	line 8:  
	
	\begin{lstlisting} 
	$amountRegex = "^\d*\.?\d*$";
	\end{lstlisting}
	
This way the check for the amount of money to transfer is server-side too. This will prevent any input of incorrect amount, even if the client-side check is bypassed.\\

 
\subsection{Test Integrity Checks(OTG-BUSLOGIC-003)}\

Fixed by fixing Injection vulnerabilities.\\
 
 
\subsection{Testing for JavaScript Execution(OTG-CLIENT-002)}\
Fixed by fixing Injection vulnerabilities. \\

 
\subsection{Testing for HTML Injection(OTG-CLIENT-003)}\
Fixed by fixing Injection vulnerabilities.\\ 

\subsection{Testing for CSS Injection(OTG-CLIENT-005)}\
Fixed by fixing Injection vulnerabilities.
\end{document}
